%%%%%%%%%%%%%%%%%%%%%%%%%%%%%%%%%%%%%%%%%
% Medium Length Graduate Curriculum Vitae
% LaTeX Template
% Version 1.1 (9/12/12)
%
% This template has been downloaded from:
% http://www.LaTeXTemplates.com
%
% Original author:
% Rensselaer Polytechnic Institute (http://www.rpi.edu/dept/arc/training/latex/resumes/)
%
% Important note:
% This template requires the res.cls file to be in the same directory as the
% .tex file. The res.cls file provides the resume style used for structuring the
% document.
%
%%%%%%%%%%%%%%%%%%%%%%%%%%%%%%%%%%%%%%%%%

%----------------------------------------------------------------------------------------
%	PACKAGES AND OTHER DOCUMENT CONFIGURATIONS
%----------------------------------------------------------------------------------------

\documentclass[]{res} % Use the res.cls style, the font size can be changed to 11pt or 12pt here
\usepackage{array}
\usepackage{wrapfig}
\usepackage{graphicx}
\usepackage{hyperref}
\usepackage[dvipsnames]{xcolor}
%\usepackage{showframe}
%\usepackage{layout}
%\usepackage{helvet} % Default font is the helvetica postscript font
%\usepackage{newcent} % To change the default font to the new century schoolbook postscript font uncomment this line and comment the one above

%\setlength{\textwidth}{5.1in} % Text width of the document

%-----------
% margins
%----------
\oddsidemargin = -50pt
\topmargin = -30pt

\textwidth = 500pt 



\begin{document}




\begin{minipage}[ht]{0.78\textwidth}
\vspace{-18pt}
\begin{wrapfigure}{l}{0.12\textwidth}
\vspace{-18pt}

    \includegraphics[width=0.7in]{BirminghamUniversityCrest.pdf}
\end{wrapfigure}
\textbf{PhD student \\
University of Birmingham \\
}
\end{minipage}
\hspace{-20pt}
\begin{minipage}[ht]{0.28\textwidth}
\Large\textbf{Coen Neijssel}\normalsize \\
phone: +31(0)614219965\\
mail: cneijssel@gmail.com\\
webpage: {\color{blue} \href{http://www.cneijssel.com}{cneijssel.com}}\\
profile: {\color{blue} \href{https://scholar.google.com/citations?hl=en&view_op=list_works&gmla=AJsN-F4qhiUM-CTnfkcztQclzMCdlI4Klho8qKf5n6g2txyW89xcNY80F_vT-CVIKH3pq1akJ7WYjbzqrtNsi690ipbzwhE-lQ&user=EO0BkhkAAAAJ}{scholar.google}}
\end{minipage}



%----------------------------------------------------------------------------------------
%	Curiculum Vitae + line
%----------------------------------------------------------------------------------------

\centerline{\Large\bf Curiculum Vitae} % Your name at the top
\centerline{\Large\bf }

{\hrule width\textwidth height 1pt}\smallskip % Horizontal line after name; adjust line thickness by changing the '1pt'


%----------------------------------------------------------------------------------------
%  Set up two column environment
%--------------------------------------------------------------------------------


\definecolor{lightgray}{gray}{0.8}
\newcolumntype{L}{>{\raggedright}p{0.14\textwidth}}
\newcolumntype{R}{p{0.8\textwidth}}
\newcommand\VRule{\color{lightgray}\vrule width 1.2pt}



% Collaborator 	      &{\textbf{---}} \\[12pt]




\begin{tabular}{L!{\VRule}R}
\Large \bf Education \normalsize & \\ [5pt]
2016--2020 &{\textbf{PhD.\ Institute for Gravitational-Wave Astronomy,}}\\
          &{\textbf{University of Birmingham, United Kingdom}}\\[1pt]
          &{A 3.5-4 year program, under the supervision of Dr.I.Mandel. The main topic is
          understanding the evolution of massive stars in binaries and how they might create binary
          black holes. I am a code developer within the COMPAS group
           (\url{https://compas.science/}. This is relvant for understanding recent detections
           of gravitational wave-events. Although based in Birmingham, I have been a long term 
            visitor at AEI (Hannover, Germany for 6 Months) and
            Monash University (Melbourne, Australia for 6 Months).}\\[5pt]

2014--2016&{\textbf{MSc.\ Astronomy and Astrophysics at the Anton Pannekoek Institute,}}\\
          &{\textbf{University of Amsterdam, Netherlands}}\\[1pt]
	  &{The program consists of 120 EC, 60 EC of astrophysical courses and a 60 EC research project. 
            Elective courses (grade) include Radio Astronomy (8/10), Particle Cosmology (8/10), and Interstellar and Circumstellar Matter (7.5/10). The research project, supervised by Selma de Mink and focused on interpreting some of the results of the VLT-Flames Tarantula Survey. }\\[5pt]


2008--2014&{\bf BSc.\ Beta-Gamma (major in Physics), University of Amsterdam, Netherlands}\\[1pt]
          &{Program consists of 180 EC, 60 EC of science and social-science courses and a 120 EC in Physics.  I chose this program because of my broad interest in science. Broader courses include; Basic Chemistry and Biology (8.5/10), The State and Society (7.5/10), Logic (8/10).}\\
	  &{University of Amsterdam, Netherlands}\\
\end{tabular}


\begin{tabular}{L!{\VRule}R}
\Large \bf Teaching \normalsize & \\ [5pt]
Co-supervisor   & {Research projects of students within COMPAS group 2016-2019}\\[5pt]

Tutor   & {Second year course in general physics at  University of Birmingham 2016-2018}\\[5pt]

Teacher & {Classes withMobile Planetarium at different schools Netherlands 2015-2016}\\[5pt]

\end{tabular}




\begin{tabular}{L!{\VRule}R}
\Large \bf Interests \normalsize & \\ [5pt]

Teaching  & {Outreach to underprivileged schools, supervision of (under-) graduate projects}\\[5pt]

Graphics  & {Conference posters, paper figures, talks}\\[5pt]

Science case   &{Evolution of massive stars, binary stars, progenitors of gravitational wave events}\\[5pt]

Techniques  & {Population Synthesis, Python, C++, and Bayesian analysis}\\[5pt]




\end{tabular}


\begin{tabular}{L!{\VRule}R}
\Large \bf Skills \normalsize & \\ [5pt]

Coding   & {Reading writing in Python, C++, Latex, reading in Fortran and basics of git workflow}\\[5pt]

Languages & {Fluent in English and Dutch, basics of Italian}\\[5pt]

Graphics  & {Basics of GIMP, Inkscape}

\end{tabular}

\newpage


{\Large{\textbf{Selection of Papers:} }\textbf{\color{blue} \href{https://arxiv.org/search/?query=Neijssel\%2C+C&searchtype=author&abstracts=show&order=-announced_date_first&size=50}{Click here to view all on ArXiv}}\\}
Per category most recent placed first

{\large \textbf{First/Second Author Papers}, }
{\color{lightgray}\hrule width 0.35\textwidth height 2pt}
\begin{enumerate}

\item \textbf{Be X-ray binaries in the SMC as indicators of mass transfer efficiency }\\
S.Vinviguerra, \textbf{C.J.\ Neijssel}, et al. (10 authors), arXiv:2003.00195, published in MNRAS

\item \textbf{The effect of the metallicity-specific star formation history on double compact object mergers}\\
\textbf{C.J.\ Neijssel}, et al. (10 authors), arXiv:1906.08136, published in MNRAS

\item \textbf{On the formation history of Galactic double neutron stars}\\
A. \ Vigna-G{\'o}mez, \textbf{C.J.\ Neijssel}, et al. (12 authors), arXiv:1805.07974, published in MNRAS
\end{enumerate}

{\large \textbf{Papers with co-authorship that used results/pipelines from paper (1) }} 
{\color{lightgray}\hrule width 0.7\textwidth height 2pt}
\begin{enumerate}
\setcounter{enumi}{2}
\item \textbf{The origin of spin in binary black holes: \\ Predicting the distributions of the main observables of Advanced LIGO}\\
S.S.Bavera, et al. (10 authors), arXiv:1906.12257
\item \textbf{The impact of pair-instability mass loss on the binary black hole mass distribution}\\
S.Stevenson, et al. (7 authors), arXiv:1904.02821, published in Astrophysical Journal
\item \textbf{Accuracy of inference on the physics of binary evolution from gravitational-wave observations}
J.W.Barrett, et al., arXiv:1711.06287, published in MNRAS
\end{enumerate}

{\large \textbf{Papers with co-authorship through discussions/code support/providing figures etc.}} 
{\color{lightgray}\hrule width \textwidth height 2pt}
\begin{enumerate}
\setcounter{enumi}{5}
\item \textbf{Luminous Red Novae: population models and future prospects}\\
G. Howitt, et al. (8 authors), arXiv:1912.07771, published in MNRAS

\item \textbf{Detecting Double Neutron Stars with LISA}\\
M.Lau, et al. (6 Authors), arXiv:1910.12422, published in MNRAS

\item \textbf{STROOPWAFEL: Simulating rare outcomes from astrophysical populations,\\ with application to gravitational-wave sources}\\
F.S.Broekgaarden, et al. (9 authors), arXiv:1905.00910, published in MNRAS

\item \textbf{saprEMo: a simplified algorithm for predicting detections of \\ electromagnetic transients in surveys}\\
S.Vinciguerra, et al. (6 authors), arXiv:1809.08641, published in MNRAS

\item \textbf{Exploring the Parameter Space of Compact Binary Population Synthesis}\\
J.W.Barrett, et al. (5 authors), arXiv:1704.03781

\item \textbf{Formation of the first three gravitational-wave observations through isolated binary evolution}\\
S.Stevenson, et al. (7 authors), arXiv:1704.01352 , published in Nature Communications
\end{enumerate}

{\large \textbf{Papers with co-authorship through efforts during my master-thesis}} 
{\color{lightgray}\hrule width 0.6\textwidth height 2pt}
\begin{enumerate}
\setcounter{enumi}{11}
\item \textbf{Predicting the Presence of Companions for Stripped-Envelope Supernovae: \\ The Case of the Broad-Lined Type Ic SN 2002ap}\\
E.Zapartas, et al. (13 authors), arXiv:1705.07898, published in Astrophysical Journal
\item \textbf{Delay-time distribution of core-collapse supernovae \\ with late events resulting from binary interaction}\\
E.Zapartas, et al. (11 authors), arXiv:1701.07032, published in Astronomy and Astrophysics
\item \textbf{The Tarantula Massive Binary Monitoring: I. \\ Observational campaign and OB-type spectroscopic binaries}\\
L.A.Almeida, et al. (27 authors), arXiv:1610.03500 , published in Astronomy and Astrophysics
\end{enumerate}




\newpage
{\Large{\textbf{Talks \& Outreach}}}

{\large{\textbf{Outreach/Teaching}}}\
{\color{lightgray}\hrule width 0.25\textwidth height 2pt}
\vspace*{7pt}
I strongly believe outreach is an important part of astrophysics and teaching in general, especially in  helping people to understand the different roles they can fulfil in science.
My favourite audience is elementary and high-school students that would otherwise not be exposed to quirky and coolness of astrophysicists. Additionally I like designing graphics and posters for outreach events. For these reasons I have:\\

 \vspace{-10pt}
\begin{itemize}
\item{Helped during Astronomy in the City events at University of Birmingham}
\item{Given talks and supported at open days at the University of Birmingham}
\item{Designed a poster for the 8th Einstein Telescope Symposium}
\item{Given classes to elementary and high-school students with the mobile planetarium
  during my master at UvA. Each session would be around 45 minutes and I would do them
  for the whole day. Instead of giving the same class each day I tried to tailor the session
  to the specific questions of the class.}
\end{itemize}

Sadly due to travelling I have not been able to help much during outreach events in the last year of my PhD.

During my PhD, I helped teaching in undergraduate courses (marking and homework sessions)
I have helped supervising several summer students and graduate students in projects within the
group of Dr.I.Mandel. I really enjoy seeing the progress they make and the confidence they acquire
during these weeks of working together.

\vspace*{20pt}

{\large{\textbf{Conferences/Workshops}}}
{\color{lightgray}\hrule width 0.3\textwidth height 2pt}
\vspace*{7pt}
\begin{itemize}
  \item \textbf{Colloquim} - 26/02/2020, Melbourne University (Australia)\\
    Discussed the evolution of massive stellar binaries for X-ray Binaries and Binary Black Holes
  \item \textbf{LIGO/Virgo sources in O3 era,} 20/01/2020 - 24/01/2020, Kavli IPMU Kashiwa, (Japan)\\
    Introduced and lead the discussion on population synthesis in the morning session
  \item \textbf{OzGrav ECR} 17/11/19 - 22/11/19, Lorne (Australia) - Poster\\
    Designed and presented the collaboration poster for COMPAS
  \item \textbf{Colloquium} - 30/08/2018 , AEI Hannover (Germany) \\
	Presented work cosmic integration and rates of Binary Black Holes 
  \item \textbf{EWASS 2018} 03/04/2018 - 06/04/2018, Liverpool (U.K.) - Talk \\
 	Presented work cosmic integration and rates of Binary Black Holes 	
  \item \textbf{ESO imbase 2017} 03/07/2017 - 07/07/2017, Garching (Germany) - Talk\\
 	Presented work on the formation and rates of Binary Black Holes
  \item \textbf{NAC} 06/11/2015, Utrecht (Netherlands) - Poster\\
 	The Dutch Astronomy conference where I presented my poster on the period distribution of massive stars.
  \item \textbf{Binary population synthesis workshop} 08/08/2015 - 11/08/2015, Cambridge (U.K.) - Short Talk\\
 	Presented my master-thesis science case for the binary\_c community for which I won a travel grant. 	

\end{itemize}


\end{document}


% 
% \large \textbf{Computer Skills} \normalsize
% \begin{itemize}
%   \item \textbf{C++, Python}, and a basic experience with \textbf{\LaTeX}, \textbf{Linux OS}, \textbf{inkscape}.
% \end{itemize}
% 
% 
% \vspace*{15pt}

% 
% \large \textbf{Working Experience} \normalsize
% \begin{itemize}
%   \item \textbf{Mobile Planetarium} 2014-2017 \\
% 	Inflatable Planetarium in which I gave lessons for children and adults about astrophysics throughout the Netherlands.\\
% \end{itemize}
